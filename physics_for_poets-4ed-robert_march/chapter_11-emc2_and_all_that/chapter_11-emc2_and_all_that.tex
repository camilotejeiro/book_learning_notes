%**********************************************************************%
%* Physics for Poets Learning Notes
%* 4th edition
%* Author: Robert March
%* Chapter: 11
%* Notes: camilo tejeiro
%**********************************************************************%

% article type 12 pt font.
\documentclass[12pt, letterpaper]{article}

%----------------------- External Packages ----------------------------%
% package to insert images to our doc.
\usepackage{graphicx}

%------------------ Dimensions and Page Layout--------------------------%

%----------------------- LaTeX Environments ---------------------------%

%------------------------- LaTeX Commands -----------------------------%

%------------------------- Document Content ---------------------------%
% make title built in command values
\title{Chapter 11, $E=mc^2$ and all that.}
\author{Learning Notes, Physics for Poets}
\date{}

\begin{document}

    \maketitle

    \section{The Meaning of $E=mc^2$}
    \textit{Ever since Hiroshima, this formula has been associated in the public 
    mind with nuclear energy. However, we must emphasize (at the risk of 
    repetition) that this formula applies equally well to all forms of 
    energy. It is a universal description of nature; as valid for a bonfire 
    as for a nuclear weapon.}   
    
    $e=mc^2$ is a statement saying that in fact for all practical purposes 
    mass and energy are identical; mass is energy and all energy has mass. 
    C is just a conversion factor from units of mass to units of energy, 
    much like converting from miles to km.
    
    \section{The Mass Increase}    
    So just as energy can change, mass is no longer a constant. 
    What we used to call mass, we must now define as the rest mass, i.e. 
    the mass of the object at rest. Because as we speed 
    up an object to relativistic speeds, its mass changes;
    from our point of view (i.e. our frame of reference) the object 
    starts to get more and more massive.

    So we introduce the symbol $m_0$ to designate the mass of an 
    object at rest (the ``constant'' mass we are familiar with) and the 
    changing or relativistic mass then becomes:
    
    $$m=\gamma m_0$$
    
    From this equation, we can see that at rest our Lorentz factor ($\gamma$) 
    is 1 and our mass is equal to our rest mass, however as our object speeds 
    up, our Lorentz factor increases and our object starts becoming more 
    massive. As our object's speed approaches that of light we see that 
    our mass approaches infinity. Thus, this \textit{relativistic mass 
    increase} enforces the speed of light as our upper boundary; our 
    maximum speed.
    
    \medskip
    \textbf{but what does this mean?}
    It means that as we do more work to increase the object's speed near 
    that of light, we are still increasing it's momentum, but our 
    energy investment is going towards increasing the object's mass while 
    the increases in speed are very small and harder and harder to achieve. 
    
    \section{Summary}    
    Despite relativistic reinterpretations of space and time, momentum conservation 
    survives if we simply allow mass to vary with velocity according to 
    the Lorentz factor.
    
\end{document}

