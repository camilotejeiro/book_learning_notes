%**********************************************************************%
%* Physics for Poets Learning Notes
%* 4th edition
%* Author: Robert March
%* Chapter: 10
%* Notes: camilo tejeiro
%**********************************************************************%

% article type 12 pt font.
\documentclass[12pt, letterpaper]{article}

%----------------------- External Packages ----------------------------%
% package to insert images to our doc.
\usepackage{graphicx}

%------------------ Dimensions and Page Layout--------------------------%

%----------------------- LaTeX Environments ---------------------------%

%------------------------- LaTeX Commands -----------------------------%

%------------------------- Document Content ---------------------------%
% make title built in command values
\title{Chapter 10, The Wedding of space and Time}
\author{Learning Notes, Physics for Poets}
\date{}

\begin{document}

    \maketitle

    There are two ways to resolve a disagreement. One is to learn to 
    \textit{live with it}, as long as all parties thoroughly understand and 
    tolerate each other's point of view. The other is to \textit{find a 
    common ground} on which all parties can agree. The theory of relativity 
    offers both of these ways to resolve the disputes between observers 
    implied by Einstein's postulate.
    
    The \textit{``live with it''} approach was outlined in Einstein's first 
    paper on special relativity, submitted to the journal \textit{Annalen der Physik} 
    in June 1905. It allows any observer to translate  the picture of reality 
    in any reference frame to that in any other; we do this by using the famous 
    Lorentz Transformations. The \textit{``common ground''} approach allows 
    us to convert our measurements to a four dimensional space-time representation 
    where we can all agree. 
    
    \section{Special Relativity, What We See}
    From our previous post on special relativity we found that our observations 
    give us a picture of reality which might differ from that of other observers, 
    mainly:
    
    \begin{itemize}
        \item[1. ] Moving clocks appear to run slow: i.e. time slows down 
        for a moving object as seen from the ground.
        \item[2. ] Moving objects appear shortened along their line of 
        motion. Objects contract along their line of motion as seen from the 
        ground.
        \item[3. ] Events that are simultaneous in one reference frame may 
        not be in another: what we consider simultaneous is really only 
        an illusion; we are not taking into account the time it takes for 
        light to reach our eyes. In fact two events that appear simultaneous 
        in a moving frame will appear to happen at different times (non-simultaneous) 
        as seen from the ground. 
    \end{itemize}    

    Einstein has, in effect, relocated the boundary between what is out there 
    in nature and what is constructed in our minds. Our observations are true 
    only by convention, in an arbitrarily chosen reference frame. Useful as 
    reference frames may be, nature doesn't hand them to us; we make them 
    up for ourselves.

    \section{Space-Time: The Fourth Dimension}
    
    When we convert our measurements to space-time coordinates, we are 
    accounting for the distance lights travels and how it affects our 
    individual measurements in different reference frames. As we move to 
    space time coordinates and compensate for relativistic effects, our 
    measurements are in agreement with each other.
    
    General relativity tells us that time can be appended to the Pythagorean 
    theorem, but in a peculiar way that reminds us that we are not dealing 
    simply with another dimension of space. Instead adding its square to 
    the squares of the space dimensions we must subtract it! If we do this, 
    we get a quantity that remains the same in all reference frames; 
    \textit{an invariant} 
    
    $$s^2=x^2 + y^2 + z^2 - (ct)^2$$

    When we move to 4 dimensional space-time all observers can agree on the 
    space-time distance between two observations.
    
    But 4 dimensional space time can not be easily visualized because it is 
    a convenient mathematical representation to account for relativistic 
    effects. 

    \section{Our View of Reality}    
    \textit{The eye can look upon objects a few inches away or gaze upon a grand 
    vista covering tens or at most hundreds of miles. But we can scarcely 
    imagine the time it takes light to cover such distances.}
    
    Our customary view of reality is like a motion picture - a series of 
    still frames showing events in different places at successive instants 
    in time. The problem is that two different movies are constructed 
    in two different reference frames. Both contain all the same events, 
    but those that are in the same still picture in one reference frame may 
    be in different pictures in another.
    
    \medskip
    Remember the difference in ``realities'' that comes about from two observers, 
    one moving at a speed close to the speed of light and the other at rest 
    (remember every observer could claim it is the other one that is moving):
    The observer at rest claims that the moving observer's clock slowed 
    down, while the moving observer claims the rest observer's clock sped 
    up, either observer knows what they saw and both realities are simply 
    made up by our brains. 
    
    While we can not deny what the other person saw in the watches, the 
    observers are free to conclude that it was the train (or terrain or 
    fabric) that contracted; by using Lorentz conversions both observers 
    can live with their realities of what they saw but gracefully disagree 
    on those things they did not see.
        
    \subsection*{Summary}
    Relativity offers two remedies for conflicting pictures in different reference 
    frames. One is to provide rules to translate from one frame to another, the 
    other constructs a new mode of representation on which all can agree. 
    
\end{document}

